\documentclass[11pt]{article}
 \setlength{\textheight}{9.4in}
\setlength{\textwidth}{6.75in}
\setlength{\hoffset}{-1.8cm}
\setlength{\voffset}{-2cm}
\title{Extension SYM de MACSYMA\\
MANUEL DE L'UTILISATEUR}
\author{Annick Valibouze\\
LITP (tour 45-55)\\
4, Place Jussieu\\
75252 Paris Cedex 05\\
Unit\'e associ\'ee au CNRS No 248\\
et\\
GDR DE CALCUL FORMEL MEDICIS\\
\small{e-mail : avb@sysal.ibp.fr}}
\date{19 juillet 1993}
\hyphenation{sen-tant puis-sance men-tai-re}
\newcommand{\indexentry}[2]{{\tt #1} & #2\\}
\begin{document}
\maketitle


\newpage
\noindent
{\bf INTRODUCTION}\\
Cette documentation concerne un module de manipulation de fonctions
sym\'etriques implant\'e en CommonLisp. Ce module, nomm\'e {\tt SYM},
se pr\'esente actuellement comme une extension du syst\`eme de calcul 
formel {\tt MACSYMA}. 

Parmi ses applications, signalons des calculs de r\'esolvantes.

\section{D\'{e}finitions et notations}
Nous consid\`ererons un anneau $\cal A$.
\subsection{Polyn\^omes sym\'etriques}
Nous nous donnons un entier $n > 0$.
\subsubsection*{Action de groupe}
Soit $S_n$ le groupe sym\'etrique de degr\'e $n$.

Soit $E$ un ensemble quelconque. Notons $E^n$ l'ensemble des
$n$-uplets d'\'el\'ements de $E$. L'action de $S_n$ sur tout $n$-uplet
$\underline a=(a_1,\ldots,a_n)$ de $E^n$ est d\'efinie comme suit~:
\begin{eqnarray*}
            S_n   \times & E^n &\longrightarrow  E^n\\
           \sigma \;\; \times & {\underline a} &  \longrightarrow 
                      \sigma{\underline a}=(a_{\sigma (1)},\ldots,a_{\sigma (n)}).
\end{eqnarray*}
Soit ${\cal A}[\underline x]$ l'anneau des polyn\^omes
en les $n$ variables
de $\underline x=(x_1,\ldots ,x_n)$ \`a coefficients dans $\cal A$. 
L'action de $S_n$ sur ${\cal A}(\underline x)$ est d\'efinie
comme suit~: 
\begin{eqnarray*}
            S_n & \times {\cal A}[\underline x] & \longrightarrow  {\cal A}[\underline x]\\
           \sigma & \times f  & \longrightarrow  \sigma f \;\; :\;
                      \sigma f({\underline x})=f(\sigma{\underline x}).
\end{eqnarray*}

{\it Remarque.} L'action de $S_n$ s'\'etend \`a des ensembles $E^c$
ou ${\cal A}[x_1,\ldots ,x_c]$ avec $c \leq n$ par simple 
plongement de ces ensembles dans $E^n$ et ${\cal A}[\underline x]$ 
respectivement.

Soit $f$ une fonction de ${\cal A}[\underline x]$ (resp. $\underline a$ un
$n$-uplet). Nous noterons $S_nf$ (resp. $S_n \underline a$) l'ensemble
$\{\sigma f \;|\; \sigma \in S_n \}$ (resp. $\{\sigma\underline a \;|\;
 \sigma
\in S_n \}$) appel\'e l'{\it orbite} de $f$ (resp. $\underline a$)
sous l'action de $S_n$.

\subsubsection*{Partitions, polyn\^omes sym\'etriques}

Soit $m$ un entier naturel.
Une {\it partition} de $m$, est une suite 
d\'ecroissante d'entiers naturels $i_1\geq i_2 \geq \ldots..$ appel\'ees 
{\it parts} dont la somme est \'egale \`a $m$. Cette somme est le {\it poids}.
de la partition et sa {\it longueur} est le nombre de ses parts non nulles.

Soit $p$ un polyn\^ome de $n$ variables. Ce polyn\^ome est dit {\it
sym\'etrique} s'il reste invariant sous l'action de $S_n$, c'est-\`a-dire si
$S_n p =\{
p\}$.

Un polyn\^ome sym\'etrique peut-\^etre repr\'esent\'e par des
partitions~: 


Soient $I=(i_1,\ldots ,i_n)$ (i.e. $i_1\geq i_2\geq \ldots \geq i_n$)
une partition de longueur inf\'erieure ou
\'egale \`a $n$ et ${\underline x}^I$ le mon\^ome 
$$
{\underline x}^I = x_1^{i_1}x_2^{i_2}\ldots x_n^{i_n}.
$$
Une {\it forme monomiale} $M_I({\underline x})$ sur ${\underline x}$ 
index\'ee par $I$ est la somme des mon\^omes de l'orbite de
${\underline x}^I$ sous l'action de $S_n$ :
$$
M_I({\underline x}) =\sum_{J \in S_nI}{\underline x}^J.
$$
Les formes monomiales constituent naturellement une base d'espace vectoriel
sur l'anneau des polyn\^omes sym\'etriques~: tout
polyn\^ome sym\'etrique se repr\'esente comme une
combinaison lin\'eaire finie de formes monomiales.

\subsubsection*{Repr\'esentations en machine}

Soit $p$ un polyn\^ome sym\'etrique sur un anneau $\cal A$.
Supposons que $p$ soit donn\'e sur la base des formes monomiales.

Dans la {\it repr\'esentation contract\'ee} de $p$
on remplace toute forme monomiale $M_I(x)$ par $x^I$, ou par tout mon\^ome
de $S_nx^I$, son orbite. 
Dans sa {\it repr\'esentation partitionn\'{e}e}, on remplace tout {\it terme
monomial}  $cM_I(x)$, o\`u $c$ est un coefficient sur $\cal A$, par la liste
$[c,i_1,\ldots ,i_n]$ et on r\'eunit le tout dans une liste.

{\it Exemple.} le polyn\^{o}me contract\'{e} associ\'{e} \`{a} 
$3x^4 + 3y^4 -2xy^5 -2x^5y$ est 
$3x^4 -2xy^5$ et le polyn\^{o}me partitionn\'{e} est
[[3,4],[-2,5,1]].\\

\subsection{Polyn\^omes multi-sym\'etriques}
Nous g\'en\'eralisons ici le cas sym\'etrique.
Nous nous donnons un entier $r >0$ et un $r$-uplet d'entiers 
$D=(d_1, \ldots ,d_r)$.

\subsubsection*{Action de groupe}

Soient $E_1, E_2, \ldots ,E_r$ $r$ ensembles quelconques.
Notons $D(E)$ le produit $E_1^{d_1}\times E_2^{d_2}\times \ldots \times E_r^{d_r}$

L'action du produit de groupes  sym\'etriques 
$S_D=S_{d_1}\times S_{d_2} \times \ldots \times S_{d_r}$ sur $D(E)$ est 
d\'efinie naturellement comme suit~:  
\begin{eqnarray*}
            S_D \times& D(E) & \longrightarrow  D(E)\\
           \sigma\;\;  \times & A  & \longrightarrow 
         \sigma A=(\sigma_1{\underline a}_1,\ldots ,\sigma_r{\underline a}_r),
\end{eqnarray*}
o\`u $\sigma = (\sigma_1,\ldots ,\sigma_r)$ avec $\sigma_i \in
S_{d_i}$. L'orbite de $A$ sous l'action de $S_D$ sera
not\'ee $S_DA$. 

De m\^eme si $X=({\underline x}_1, \ldots,{\underline x}_r)$ est un 
$r$-uplet tel que la $i$-i\`eme composante
${\underline x}_i$ est un $d_i$-uplet de variables, on notera ${\cal A}[X]$
l'ensemble des polyn\^omes en les variables de $X$ et \`a coefficients dans 
$\cal A$. 
L'action de $S_D$ sur ${\cal A}[X]$ est d\'efinie
comme suit~:
\begin{eqnarray*}
            S_D & \times {\cal A}[X] & \longrightarrow {\cal A}[X]\\
           \sigma & \times f  & \longrightarrow \sigma f \;\; :\;
         \sigma f(X)= f(\sigma X).
\end{eqnarray*}
L'orbite de $f$ sous l'action de $S_D$ sera
not\'ee $S_Df$.

\subsubsection*{Multi-partitions, Polyn\^omes multi-sym\'etriques}

Une {\it multi-partition}, $I$, d'ordre $r$ est un $r$-uplet de partitions.
Nous appellerons la {\it multi-longueur} de $I$ la liste des longueur
des partitions qui constituent $I$.

Soit $p$ un polyn\^ome de $d_1+\cdots + d_r$ variables. Ce polyn\^ome
est dit {\it multi-sym\'etrique} s'il reste invariant par $S_D$,
c'est-a-dire si $S_D p =\{p\}$.

Un polyn\^ome multi-sym\'etrique peut-\^etre repr\'esent\'e par des
multi-partitions~:

Soient $X$ comme ci-dessus et $I=(I_1,\ldots,I_r)$ un 
$r$-uplet de ${\bf N}^{d_1}\times \ldots \times {\bf N}^{d_r}$.
on notera naturellement par $X^I$ le mon\^ome 
$$
X^I = {\underline x}_1^{I_1}\ldots{\underline
x}_r^{I_r}. 
$$
Si $I$ est une multi-partition de multi-longueur inf\'erieure \`a $D$,
nous d\'efinissons la
{\it multi-forme monomiale} $M_I(X)$ sur $X$ 
index\'ee par $I$ par la somme des mon\^ome de l'orbite de
$X^I$ sous l'action de $S_D$ :
$$
M_I(X) =\sum_{J \in S_DI} X^J.
$$

Il est naturel de repr\'esenter un polyn\^ome multi-sym\'etrique par une
combinaison lin\'eaire de multi-forme monomiales sur l'anneau des
coefficients du polyn\^ome.
\subsubsection*{R\'epresentation machine}
Consid\'erons maintenant un polyn\^ome multi-sym\'etrique \`a
coefficients dans un anneau $\cal A$ en les variables de $X$. 
Ce polyn\^ome s'exprime naturellement comme
une somme finie de $cM_I(X)$ o\`u $c \in {\cal A}$ et $M_I(X)$ est une
multi-forme monomiale sur $X$. Une {\it forme contract\'ee}
associ\'ee \`a un polyn\^ome multi-sym\'etrique
consiste \`a remplacer dans ce polyn\^ome chaque $M_I(X)$ par le mon\^ome 
$X^I$ ou bien par un autre mon\^ome de son orbite $S_DX^I$.

\section{Initialisation et mode op\'eratoire}

Le fichier sym.mac comporte toutes les commandes d'auto-chargement. C'est
\'eventuellement dans ce fichier qu'il faut modifier des noms de chemins si
l'utilisation de {\tt SYM} doit se faire d'un autre directory que celui
o\`u il est install\'e.

le fichier {\tt compile.lsp} comporte la liste des fichiers \`a
compiler. Pour l'ex\'ecuter on peut le charger sous {\tt LISP} ou sous
{\tt MACSYMA}
par la commande {\tt load(``compile.lsp'').} 
Une fois les fichiers de {\tt SYM} compil\'es on charge le module sous
{\tt MACSYMA} avec le fichier {\tt sym.mac}~: 
\begin{verbatim}
load(``sym.mac'');
\end{verbatim}

Sauf en cas de pr\'{e}cision, les fonctions appel\'{e}es compl\`{e}tent les
listes avec des valeurs formelles.
Pour la i\`{e}me fonction sym\'{e}trique \'{e}l\'{e}mentaire, ce sera {\tt ei},
pour la i\`{e}me fonction puissance, ce sera {\tt pi}, et pour la i\`{e}me 
fonction compl\`ete  ce sera {\tt hi}.\\

Il existe plusieurs modes d'\'evaluation de polyn\^omes sous MACSYMA :
meval, expand, rat, ratsimp.
SYM permet le choix du mode op\'eratoire.
A chaque appel d'une fonction, SYM teste si le drapeau de la variable {\tt oper}
a \'et\'e modifi\'e. Dans ce cas, le mode op\'eratoire est modifi\'e ainsi~:
\begin{description}
\item Si {\tt oper = meval} (par d\'efaut), on fait
      les op\'erations avec {\tt meval} .
\item Si {\tt oper = expand}, elles sont faites avec {\tt expand}.
      
\item Si {\tt oper = rat}, elles sont faites avec {\tt rat}.
\item Si {\tt oper = ratsimp}, elles sont faites avec {\tt ratsimp}.

\end{description}
Le mode {\tt meval} est avantageux en num\'erique. Avec des valeurs
formelles le mode {\tt rat} est souvent pr\'ef\'erable.

Pour mettre le drapeau d'{\tt oper} \`a {\tt expand}, par exemple, il
suffit d'\'ecrire :
\begin{center}
{\tt oper : expand;}
\end{center}
\section{Descriptif des fonctions disponibles}

Prenons q la valeur minimale entre le degr\'{e} du polyn\^ome {\tt sym}
 et le cardinal {\tt card}.
On met dans {\tt lvar} les variables du polyn\^ome consid\'er\'e le
distinguant ainsi des param\`etres \'eventuels.

\subsection{Combinatoire}
  \begin{itemize}
    \item {\tt ARITE(DEGRE, ARITE, PUISSANCES)}
\index{ARITE(DEGRE, ARITE, PUISSANCES)}
applique le the'ore`me de l'arite'
(A. Valibouze (1992) Sur l'arit\'e des fonctions, \`a para{\^ \i}tre dans 
la revue European Journal of Combinatorics ). Cette fonction permet de
passer d'une fonction puissance d'une  resolvante en ARITE variables 
a une fonction puissance en DEGRE variables. Elle rajoute un
coefficient binomial a chaque partition. On suppose que les fonctions
puissances sont donne\'es sur la base des formes monomiales de mani\`ere
partitionn\'ee dans la liste PUISSANCES.

    \item {\tt CARD\_ORBIT(partition,n)}
\index{CARD\_ORBIT(partition,n)}
 $\longrightarrow$ {\tt entier}\\
 {\tt partition} est une partition donn\'ee sous la forme 
$[a_1,m_1,...,a_q,m_q]$ o\`u $m_i$ est la multiplicit\'e de $a_i$
dans la partition.
La fonction calcule le cardinal de l'orbite de la partition sous
l'action du groupe sym\'etrique de degr\'e {\tt n}. 

    \item {\tt MULTINOMIAL(r,part)} \index{MULTINOMIAL(r part)}
 $\longrightarrow$ {\tt entier}\\
 o\`u {\tt r} est le poids de la partition {\tt part}. Cette
fonction ram\`ene le coefficient multinomial associ\'e : si les
parts de la partitions {\tt part} sont $i_1, i_2, ..., i_k$, le r\'esultat de
{\tt MULTINOMIAL} est $r!/(i_1!i_2!...i_k!)$.

    \item {\tt CARD\_STAB(l,egal)}
\index{CARD\_STAB(l,egal)}  $\longrightarrow$ {\tt entier}\\
{\tt l} est une liste d'objets ordonn\'es et {\tt egal} est le
test d'\'egalit\'e entre eux. Soit n la longueur de la liste {\tt l}.
Cette  fonction calcule le cardinal du stabilisateur de {\tt l} sous
l'action  du groupe sym\'etrique d'ordre n.

 \item {\tt PERMUT(l)} \index{PERMUT(l)}
 $\longrightarrow$ liste\\
 ram\`ene la liste des permutations de la liste {\tt l}.

\end{itemize}
\small
\begin{verbatim}
 CARD_ORBIT([5,2,1,3],6);
                                 60
 CARD_STAB([a, a, c, b, b], eq);
                                  4
 CARD_STAB([1,1,2,3,3], "=");
                                  4

  arite(4,2,[[[1,2,4]],[[1,5,5],[1,2]],[[1,2,2],[1,3]]]);

       [[[1, 2, 4]], [[1, 5, 5], [3, 2]], [[1, 2, 2], [3, 3]]]
\end{verbatim}
Ci-dessus la liste des
fonctions puissances est $[x^2y^4,x^5y^5 + x^2,x^2y^2+x^3]$ l'arit\'e 
est 2 et le degr\'e est 4. Cette fonction est utile aux calculs de 
r\'esolvantes.
\normalsize
\subsection{Sur les polyn\^omes d'une variable}
  \begin{itemize}
    \item {\tt ELE2POLYNOME(cele,z)} \index{ELE2POLYNOME(cele,z)}
$\longrightarrow$ polyn\^ome\\
donne le polyn\^ome en {\tt z} dont les fonctions
sym\'etriques \'el\'ementaires des racines sont dans la liste {\tt cele}.
{\tt cele}=$[n,e_1,...,e_n]$ o\`u $n$ est le degr\'e du polyn\^ome et
$e_i$ la $i$-i\`eme fonction sym\'etrique \'el\'ementaire.

     \item {\tt POLYNOME2ELE(polyn\^ome,z)} \index{POLYNOME2ELE(polyn\^ome,z)}
$\longrightarrow$ {\tt cele}\\
 donne la liste {\tt cele}=$[n,e_1,...,e_n]$ o\`u $n$ est le
degr\'e du polyn\^ome de la variable {\tt z}
et $e_i$ la $i$-i\`eme fonction sym\'etrique 
\'el\'ementaire des ses racines.
\end{itemize}
\small
\begin{verbatim}
 ELE2POLYNOME([2,e1,e2],z);
                                  2
                                 Z  - E1 Z + E2

 POLYNOME2ELE(X^7-14*X^5  + 56*X^3  - 56*X + 22,X);
 
                      [7, 0, - 14, 0, 56, 0, - 56, - 22] 

 ELE2POLYNOME( [7, 0, - 14, 0, 56, 0, - 56, - 22],X);

                          7       5       3
                         X  - 14 X  + 56 X  - 56 X + 22
\end{verbatim}
\normalsize
\subsection{Changements de repr\'esentations}
Un polyn\^ome sym\'etrique peut \^etre donn\'e sous plusieurs formes :
\'eetendue, contract\'ee, partitionn\'ee. Les fonctions d\'ecrites ci-dessous
permettent de passer d'une forme \`a une autre. Certaines r\'ealisent de plus
un test de sym\'etrie.

Dans tout les cas les variables du polyn\^ome sont contenues dans la liste
{\tt lvar}.
\begin{itemize}
\item {\tt TPARTPOL(polyn\^ome,lvar) \index{tpartpol(polyn\^ome,lvar)}
$\longrightarrow$ ppart ,\\
       PARTPOL(polyn\^ome, lvar) \index{partpol(polyn\^ome, lvar)}
$\longrightarrow$ ppart}\\
 ram\`{e}nent, ordonn\'e dans l'ordre lexicographique croissant et
d\'ecroissant respectivement, le
polyn\^{o}me partitionn\'{e} associ\'{e} au polyn\^{o}me donn\'e sous sa forme
\'etendue. Si le polyn\^ome n'est pas sym\'etrique, la fonction
{\tt TPARTPOL} d\'eclenche une erreur.

\item {\tt TCONTRACT(polyn\^ome,lvar)\index{tcontract(polyn\^ome,lvar)}
 $\longrightarrow$ pc}\\
 {\tt CONTRACT(polyn\^ome,lvar)\index{contract(polyn\^ome,lvar)}
 $\longrightarrow$ pc}\\
agissent respectivement comme {\tt TPARTPOL} et{\tt PARTPOL} en restituant la repr\'esention contract\'ee.

\item  {\tt CONT2PART(pc,lvar) \index{cont2part(pc,lvar)}
 $\longrightarrow$ ppart }\\
rend le polyn\^ome partitionn\'e {\tt ppart} d'un polyn\^ome sym\'etrique
donn\'e sous une forme contract\'ee {\tt pc}.

\item {\tt PART2CONT(ppart,lvar) \index{part2cont(ppart,lvar)}
 $\longrightarrow$ pc }\\
rend une forme contract\'ee {\tt pc} d'un polyn\^ome sym\'etrique
donn\'e sous sa forme partitionn\'e {\tt ppart}.

\item  {\tt EXPLOSE(pc,lvar) \index{explose(pc,lvar) }
$\longrightarrow$ psym }\\
rend la forme \'etendue {\tt psym} d'un polyn\^ome sym\'etrique
donn\'e sous une forme contract\'ee {\tt pc}.

\end{itemize}
\small
\begin{verbatim}
 TPARTPOL(x^3-y,[x,y]);

                         manque des monomes
\end{verbatim}
Soit le  polyn\^ome sym\'etrique de ${\bf Z}[x,y,z]$ dont une forme 
contract\'ee est $2a^3bx^4y$. Nous allons lui imposer divers changements
de repres\'entations.

\small
\begin{verbatim}
 lvar : [x,y,z]$
 pc : 2*a^3*b*x^4*y;  
                                3    4
                             2 a  b x  y
 psym : EXPLOSE(pc,lvar);
                         3      4      3      4      3    4   
                      2 a  b y z  + 2 a  b x z  + 2 a  b y  z 
             3    4        3      4      3    4
        + 2 a  b x  z + 2 a  b x y  + 2 a  b x  y
\end{verbatim}
\normalsize
Et si on lui applique la fonction {\tt CONTRACT} on retrouve une forme 
contract\'ee~:
\small
\begin{verbatim}
 CONTRACT(psym,lvar);
                                 3    4
                              2 a  b x  y
 TCONTRACT(psym,lvar);
                                 3    4
                              2 a  b x  y
 TPARTPOL(psym,lvar);
                                  3
                             [[2 a  b, 4, 1]]
 PARTPOL(psym,lvar);
                                  3
                             [[2 a  b, 4, 1]]
 ppart : CONT2PART(pc,lvar);
                                  3
                             [[2 a  b, 4, 1]]
 PART2CONT(ppart,lvar);
                                 3    4
                              2 a  b x  y
\end{verbatim}
\normalsize

\subsection{Les fonctions li\'ees aux partitions}
\begin{itemize}
\item {\tt KOSTKA(part1,part2)} \index{KOSTKA(part1,part2)}
$\longrightarrow$ entier\\
 (\'ecrit par P.ESPERET) ram\`ene le
nombre de Kostka associ\'e aux partitions {\tt part1} et {\tt part2}.
\item {\tt TREINAT(part)} \index{treinat(part)} 
$\longrightarrow$ liste des partitions 
inf\'erieures pour l'ordre naturel
\`a la partition {\tt part} et de m\^eme poids.
\item{\tt TREILLIS(n)} \index{treillis(n)} 
$\longrightarrow$ liste des partitions de poids $n$.
\item {\tt LGTREILLIS(n,m)} \index{lgtreillis(n,m)} 
$\longrightarrow$ 
liste des partitions de poids $n$ et
de longueur  $m$.
\item {\tt LTREILLIS(n,m)} \index{ltreillis(n,m)} 
$\longrightarrow$ 
liste des partitions de poids $n$ et
de longueur inf\'erieure ou \'egale \`a  $m$.
\end{itemize}
\small
\begin{verbatim}
 KOSTKA([3,3,3],[2,2,2,1,1,1]);
                                  6
 LGTREILLIS(4,2);
                           [[3, 1], [2, 2]]
 LTREILLIS(4,2);
                         [[4, 0], [3, 1], [2, 2]]
 TREILLIS(4);
                [[4], [3, 1], [2, 2], [2, 1, 1], [1, 1, 1, 1]]
 TREINAT([5]);
                               [[5]]
 TREINAT([1,1,1,1,1]);
                        [[5], [4, 1], [3, 2], [3, 1, 1],
                    [2, 2, 1], [2, 1, 1, 1], [1, 1, 1, 1, 1]]
 TREINAT([3,2]);
                        [[5], [4, 1], [3, 2]]
\end{verbatim}
\normalsize
\subsection{Les calculs d'orbites}
\begin{itemize}
\item {\tt ORBIT(polyn\^ome,lvar) \index{orbit(polyn\^ome,lvar) }
$\longrightarrow$ $S_n$(polyn\^ome) \\}
ram\`ene  la liste des polyn\^omes de l'orbite du polyn\^ome
sous l'action
du groupe sym\'etrique $S_n$. Les $n$ variables du polyn\^ome
variables sont contenues dans la liste {\tt lvar}.
\item {\tt MULTI\_ORBIT(polyn\^ome,[lvar$_{1}$,lvar$_{2}$,\ldots ,lvar$_{r}$])
\index{multi\_orbit(polyn\^ome,[lvar$_{1}$,lvar$_{2}$,\ldots ,lvar$_{r}$])}
$\longrightarrow$ ${S_D}$(polyn\^ome) }\\
les variables de {\tt polyn\^ome} sont dans les listes de variables
{\tt lvar$_1$,lvar$_{2}$,\ldots ,lvar$_{r}$} sur
lesquelles on fait agir respectivement les groupes sym\'etriques
$S_{d_1},S_{d_2},\ldots ,S_{d_r}$. Cette fonction
ram\`ene l'orbite du polyn\^ome sous l'action du produit
$S_D$ de ces groupes sym\'etriques.
\end{itemize}
\small
\begin{verbatim}
 ORBIT(a*x+b*y,[x,y]);
                            [a y + b x, b y + a x]
 ORBIT(2*x+x**2,[x,y,z]);
                                       2         2         2
                                     [z  + 2 z, y  + 2 y, x  + 2 x]
 MULTI_ORBIT(a*x+b*y,[[x,y],[a,b]]);
                                        [b y + a x, a y + b x]
 MULTI_ORBIT(x+y+2*a,[[x,y],[a,b,c]]);
 
                  [y + x + 2 c, y + x + 2 b, y + x + 2 a]
\end{verbatim}
\normalsize
\subsection{Le produit contract\'e de deux polyn\^omes sym\'etriques}
\begin{itemize}
\item{\tt MULTSYM(ppart1, ppart2,n)\index{multsym(ppart1, ppart2,n)}
  $\longrightarrow$ pc} \\
calcule le produit de deux polyn\^omes
sym\'etriques de {\tt n} variables dont {\tt part1} et {\tt part2} sont les
formes partitionn\'ees associ\'ees.

\end{itemize}
Soient deux polyn\^omes sym\'etriques {\tt p1} et {\tt p2}. On va
calculer le produit des polyn\^omes par une m\'ethode classique du 
produit de deux polyn\^omes quelconques, puis on r\'ealisera
ce produit avec {\tt MULTSYM}. On se place dans ${\bf Z}[x,y]$.
\small
\begin{verbatim}
 
 p1 : x*y^2  + x^2*y$
 p2 : y+x$
 prod : expand(p1*p2);
                              3      2  2    3
                           x y  + 2 x  y  + x  y
\end{verbatim}
\normalsize
Constatons c'est la forme \'etendue du produit obtenu avec {\tt MULTSYM}~:
\small
\begin{verbatim}
 PARTPOL(prod,[x,y]);
                                [[1, 3, 1], [2, 2, 2]]
 ppart1 : PARTPOL(p1,[x,y]);
                                   [[1, 2, 1]]
 ppart2 : PARTPOL(p2,[x,y]);
                                   [[1, 1, 0]]
 MULTSYM(ppart1, ppart2, 2);
                                [[1, 3, 1], [2, 2, 2]]
\end{verbatim}
\normalsize

\subsection{Les changements de bases}
De mani\`ere g\'en\'erale, la liste {\tt lvar} repr\'ente la liste des
variables des polyn\^omes.
\begin{itemize}
\item {\tt ELEM(cele,sym,lvar) \index{elem(cele,sym,lvar)}
  $\longrightarrow$  P(e1,..., eq)}\\
 d\'{e}compose le polyn\^{o}me sym\'{e}trique {\tt sym} 
en les fonctions  sym\'{e}triques \'{e}l\'{e}mentaires
contenues dans {\tt cele} (3 drapeaux possibles).
\item {\tt multi\_elem([cele$_{1}$, \ldots, cele$_{r}$],multi\_pc,
[lvar$_{1}$, \ldots,lvar$_{r}$])  
\index{MULTI\_ELEM([cele$_{1}$, \ldots, cele$_{p}$],multi\_pc,
[lvar$_{1}$, \ldots,lvar$_{r}$])  }
$\longrightarrow$ 
 P(cele$_{1}$, \ldots ,cele$_{r}$)}\\
On a un polyn\^{o}me {\tt multi\_pc} multi-sym\'{e}trique sous l'action
de $S_D$ donn\'e par sa
forme multi-contract\'ee.
On le d\'ecompose
successivement en chacun des groupes {\tt cele$_{j}$} de
fonctions  sym\'{e}triques \'{e}l\'{e}mentaires de l'ensemble 
${\underline x}_j$. La liste
de variables {\tt lvar}$_j$ permet de lire le polyn\^ome. 
On y trouve les variables de ${\underline x}_j$ intervenant dans l'expression 
du polyn\^ome multi-contract\'e.

\item {\tt PUI(cpui,sym,lvar) \index{pui(cpui,sym,lvar) }
$\longrightarrow$ P(p1, ... ,pq)}\\
 d\'{e}compose un polyn\^{o}me sym\'{e}trique en les 
fonctions
puissances (3 drapeaux possibles).
\item {\tt multi\_pui([cpui$_{1}$, \ldots, cpui$_{p}$],multi\_pc,
[lvar$_{1}$, \ldots,lvar$_{p}$]) 
\index{multi\_pui([cpui$_{1}$, \ldots, cpui$_{p}$],multi\_pc,
[lvar$_{1}$, \ldots,lvar$_{p}$]) }
 $\longrightarrow$ 
 P(cpui$_{1}$, \ldots ,cpui$_{p}$ )}\\
agit comme {\tt MULTI\_ELEM} en d\'ecomposant le polyn\^ome multi-sym\'etrique
sur les fonctions puissances.
\end{itemize}

Si le polyn\^{o}me sym\'{e}trique est sous une forme contract\'{e}e
le drapeau {\tt elem} doit \^{e}tre \`{a} 1 (sa valeur par d\'{e}faut).

Consid\'erons le polyn\^ome sym\'etrique 
$x^4+y^4+z^4+t^4 - 2*(x*(y + z +t) + y*(z + t) + z*t)$
dont une forme contract\'ee est $x^4 -2*y*z$.
\small
\begin{verbatim}
elem:1$

elem([],x**4 - 2*y*z, [x,y,z]); 

                   4          2                 2
                 e1  - 4 e2 e1  + 4 e3 e1 + 2 e2  - 2 e2 - 4 e4
\end{verbatim}
\normalsize
Supposons maintenant que le nombre de variables du polyn\^ome
sym\'etrique est 3 (i.e. $t=0$). Cette valeur 3 doit \^etre signal\'ee
en t\^ete de la liste {\tt cele} comme suit~:
\small
\begin{verbatim}
elem([3],x**4 - 2*y*z,[x,y,z]);

                      4          2                 2
                    e1  - 4 e2 e1  + 4 e3 e1 + 2 e2  - 2 e2
\end{verbatim}
\normalsize
Si de plus la premi\`ere fonction sym\'etrique \'elementaire
a une valeur particuli\`ere, $e_1=7$, on la met en deuxi\`eme
\'el\'ement de la liste {\tt cele}~:
\small
\begin{verbatim}
  ELEM([3,7],x^4-2*x*y,[x,y]);

                               2
                   28 e3 + 2 e2  - 198 e2 + 2401
\end{verbatim}
\normalsize
Le $i$+1-i\`eme \'el\'ement de la liste cele doit \^etre la $i$-i\`eme
fonction sym\'etrique \'el\'ementaire.

Si le  polyn\^{o}me sym\'{e}trique est sous une forme \'etendue
le drapeau {\tt elem} doit \^{e}tre \`{a} 2~ et s'il est 
sous une forme partitionn\'ee le drapeau {\tt elem} doit \^{e}tre \`{a} 3~:\\

\small
\begin{verbatim}
 elem :2$
 ELEM([3,f1,f2,f3],x^4+y^4+z^4 - 2*(x*y + x*z + y*z),[x,y,z]);

                      4          2                 2
                    f1  - 4 f2 f1  + 4 f3 f1 + 2 f2  - 2 f2

 elem:3$
 ELEM(([],[[1, 2, 1]],[]);
                             E1 E2 - 3 E3
\end{verbatim}
\normalsize

Il en est de m\^{e}me pour le drapeau {\tt pui} associ\'e \`a la fonction
{\tt PUI}.\\

Pour la fonction {\tt PUI}, si des valeurs formelles doivent
\^{e}tre rajout\'{e}es \`{a}
{\tt cpui} on tient compte du cardinal de l'alphabet, s'il est
fournit, pour calculer les fonctions puissances en fonction des 
premi\`{e}res. On se sert alors de la fonction {\tt PUIREDUC} (voir
plus loin).
\small
\begin{verbatim}
  MULTI_ELEM([[2,e1,e2],[2,f1,f2]],a*x+a^2+x^3,[[x,y],[a,b]]);

                                2                       3
                     - 2 f2 + f1  + e1 f1 - 3 e1 e2 + e1

  MULTI_PUI([[2,p1,p2],[2,t1,t2]],a*x+a^2+x^3,[[x,y],[a,b]]);
    
                                              3
                                       3 P1 P2   P1
                          T2 + P1 T1 + ------- - ---
                                          2       2
\end{verbatim}
\normalsize
\begin{itemize}
\item {\tt ELE2PUI(m,cele) \index{ele2pui(m,cele) }
$\longrightarrow$ cpui}\\
 r\'{e}alise le passage des fonctions 
sym\'{e}triques 
\'{e}l\'{e}mentaires aux fonctions puissances de 1 \`{a} m.

\item {\tt PUI2ELE(n,cpui) \index{pui2ele(n,cpui) }
$\longrightarrow$ cele}\\
 r\'{e}alise le passage des fonctions 
puissances aux fonctions 
sym\'{e}triques \'{e}l\'{e}mentaires. Si le drapeau {\tt pui2ele} est
\'egal \`a 
{\tt girard}, on r\'ecup\`ere la liste des fonctions sym\'etriques
\'el\'ementaires de 1 \`{a} {\tt n}, et s'il est \'egal \`a {\tt close}, on
r\'ecup\`ere la {\tt n}$^{i\grave{e}me}$ fonction sym\'etrique
\'el\'ementaire.
\end{itemize}
Ci-dessous, on cherche les 3 premi\`{e}res fonctions sym\'{e}triques 
\'{e}l\'{e}mentaires.
   On ne donne pas de valeur aux 3 premi\`{e}res fonctions puissances. La
   fonction rajoute donc des variables formelles.
\small
\begin{verbatim}
 PUI2ELE(3,[]);

                           2                      3
                         p1    p2  p3   p1 p2   p1
                 [3, p1, --- - --, -- - ----- + ---]
                          2    2   3      2      6


\end{verbatim}
\normalsize
Ci-dessous on cherche les 4 premi\`eres fonctions sym\'etriques 
\'el\'ementaires en fonctions des fonctions puissances telles
que $p_1=2$. Le cardinal
de l'alphabet \'etant 3, la quatri\`eme fonction sym\'etrique
\'el\'ementaire est donc nulles. Ensuite la fonction {\tt ELE2PUI}
calcule les 3 premi\`res fonctions pussance en fonctions des 3 premi\`res 
fonctions sym\'etriques \'el\'ementaires.
\small
\begin{verbatim}
 PUI2ELE(4,[3,2]);
                         4 - p2  p3 - 3 p2 + 4
                 [3, 2 , ------, -------------, 0]
                           2        3   
 ELE2PUI(3,[]);
                             2                            3
                   [3, e1, e1  - 2 e2, 3 e3 - 3 e1 e2 + e1 ]
\end{verbatim}
\normalsize
Ici, comme le cardinal est 2, la troisi\`{e}me fonction sym\'{e}trique
\'{e}l\'{e}mentaire est nulle~:
\small
\begin{verbatim}
 ELE2PUI(3,[2]);
                                2           3
                      [2, e1, e1  - 2 e2, e1  - 3 e1 e2]
\end{verbatim}
\normalsize
\begin{itemize}
\item {\tt PUIREDUC(n,cpui) \index{puireduc(n,cpui) }
$\longrightarrow$ [card,$p_{1},p_{2},p_{3},...,p_{n}$]}\\ 
permet d'avoir 
les fonctions puissances jusqu'\`{a} {\tt n} connaissant celles jusqu'\`{a} 
{\tt m}. Le cardinal est pr\'{e}cis\'{e} dans {\tt cpui}.
\end{itemize}

Si le cardinal, n, de l'alphabet est donn\'{e}, on peut
exprimer toutes les fonctions puissances en fonction des n premi\`{e}res.
On prend par exemple 2 pour cardinal et on demande les 3 premi\`{e}res
fonctions puissances.
\small
\begin{verbatim}
 PUIREDUC(3,[2]);
                                       3
                           3 p1 p2   p1
               [2, p1, p2, ------- - ---]
                             2        2

\end{verbatim}
\normalsize
\begin{itemize}
\item {\tt ELE2COMP(m , cele) \index{ele2comp(m , cele)}
 $\longrightarrow$ ccomp}\\
 r\'{e}alise le passage des fonctions 
sym\'{e}triques 
\'{e}l\'{e}mentaires aux fonctions sym\'etriques compl\`etes de 1 \`{a} m.
\item {\tt PUI2COMP(n, cpui) \index{pui2comp(n, cpui) }
$\longrightarrow$ ccomp}\\
 r\'{e}alise le passage des fonctions 
puissances aux fonctions 
sym\'{e}triques compl\`etes de 1 \`{a} n.
\item {\tt COMP2ELE(n, ccomp) \index{comp2ele(n, ccomp) }
$\longrightarrow$ cele}\\
 r\'{e}alise le passage des fonctions sym\'etriques
compl\`etes aux fonctions 
sym\'{e}triques \'{e}l\'{e}mentaires de 1 \`{a} n.
\item {\tt COMP2PUI(n, ccomp)\index{comp2pui(n, ccomp)}
 $\longrightarrow$ cpui}\\
 r\'{e}alise le passage des fonctions sym\'etriques
compl\`etes aux fonctions 
puissances de 1 \`{a} n.
\item {\tt MON2SCHUR(liste) \index{mon2schur(liste) }
$\longrightarrow$ pc}\\
r\'ealise le passage des formes monomiales aux fonctions de Schur. Le
r\'esultat {\tt pc} est donc un polyn\^ome sym\'etrique donn\'e sous
une forme contract\'ee.

\item {\tt SCHUR2COMP(P,[h$i_{1}$,...,h$i_{q}$]))
 \index{schur2comp(P,l) }
$\longrightarrow$ liste de listes}\\
r\'ealise le passage des fonctions de Schur, not\'ees $S_{I}$,
aux fonctions totales. Le polyn\^ome P est un polyn\^ome en les fonctions
totales h$i_{k}$. Il est indispensable de noter ces fonctions totales
avec un h concat\'en\'e \`a un entier.

\end{itemize}

La fonction {\tt MON2SCHUR} permet d'\'ecrire une fonction
de Schur sur la base des formes monomiales 
repr\'esent\'ees sous leur forme contract\'ee. La fonction de Schur
est donn\'ee par une partition. Nous allons d'abord 
v\'erifier que la fonction de Schur associ\'e \`a la
partition $(1^3)$ est \'egale \`a la $3^{i\grave{e}me}$
fonction sym\'etrique \'el\'ementaire et que celle associ\'ee
\`a la partition $(3)$ est \'egale \`a la $3^{i\grave{e}me}$ fonction
sym\'etrique compl\`ete (ceci d\'ecoule d'un r\'esultat g\'en\'eral).
\small
\begin{verbatim}

 MON2SCHUR([1,1,1]);
                       x1 x2 x3
 MON2SCHUR([3]);
                           2        3
               x1 x2 x3 + x1  x2 + x1

 MON2SCHUR([1,2]);
                                   2
                    2 x1 x2 x3 + x1  x2
\end{verbatim}
\normalsize
Voyons sur un exemple comment, en op\'erant circulairement sur des changements
de bases, on recup\'ere bien sur la donn\'ee initiale.
\small
\begin{verbatim}
a1 :  PUI2COMP(3,[3]);
                                2                 3
                         p2   p1   p3   p1 p2   p1
                 [3, p1, -- + ---, -- + ----- + ---]
                          2    2   3      2      6
a2 : COMP2ELE(3, a1);
                      2                      3
                    p1    p2  p3   p1 p2   p1
            [3, p1, --- - --, -- - ----- + ---]
                     2    2   3      2      6
a3 : ELE2PUI(3,a2);
                       [3, p1, p2, p3]

a4 : COMP2PUI(3,[]);
                          2                     3
         [3, h1, 2 h2 - h1 , 3 h3 - 3 h1 h2 + h1 ]

a5 : PUI2ELE(3,a4);
                     2                        3
           [3, h1, h1  - h2, h3 - 2 h1 h2 + h1 ]

a6 : ELE2COMP(3,a5);
                       [3, h1, h2, h3]
\end{verbatim}
\normalsize
Ci-dessous on montre comment exprimer une fonction de Schur sur les
bases des formes monomiales (en {\tt c48}), des fonctions compl\`etes
(en {\tt c50}), des fonctions sym\'etriques \'el\'ementaires (en {\tt c51})
et des fonctions puissances (en {\tt c52}).
\small
\begin{verbatim}
(c48)  MON2SCHUR([1,2]);

                           2
(d48)       2 x1 x2 x3 + x1  x2

(c49) COMP2ELE(3,[]);

                     2                        3
(d49)      [3, h1, h1  - h2, h3 - 2 h1 h2 + h1 ]

(c50) ELEM(d49,d48,[x1,x2,x3]);

(d50)             h1 h2 - h3

(c51) ELEM([],d48,[x1,x2,x3]);

(d51)             e1 e2 - e3

(c52) PUI([],d48,[x1,x2,x3]);

                3
              p1    p3
(d52)         --- - --
              3    3

(c53) SCHUR2COMP(h1*h2-h3,[h1,h2,h3]);


(d53) 				    s	  
				     1, 2

(c54) SCHUR2COMP(a*h3,[h3]);

(d54)                                 s  a
                                       3

\end{verbatim}
\normalsize
On a, en derni\`eres instructions, d\'ecompos\'e $h_{1}h_{2}-h_{3}$ 
et $h_{3}$ sur la base des fonctions de Schur.
\subsection{Les r\'esolvantes}
  Soit $p$ un polyn\^ome d'un variable $x$ et de degr\'e $n$ sur un
anneau $A$.
Soit $f \in A[x_1,x_2,\ldots ,x_n]$ une fonction de transformation et
$S_nf$ l'orbite de la fonction $f$ sur l'action du groupe sym\'etrique
$S_n$. Alors la r\'esolvante de $p$ par $f$, not\'ee $f_*(p)$,
est le polyn\^ome unitaire :
\begin{eqnarray*}
f_*(p)(y) = \prod_{h\in S_nf} (y- h(\alpha_1,\ldots ,\alpha_n)),
\end{eqnarray*}
o\`u $\alpha_1,\ldots ,\alpha_n$ sont les racines de $p$.\\

Si la fonction de transformation est 
\begin{enumerate}
\item un polyn\^ome d'une variable ,
\item un polyn\^ome lin\'eaire,
\item une polyn\^ome lin\'eaire dont les coefficients non nuls sont altern\'es,
\item une polyn\^ome lin\'eaire avec que des coefficients unitaires,
\item un polyn\^ome sym\'etrique en les variables qui apparaissent dans son
expression,
\item un mon\^ome dont le coefficient est 1,
\item la fonction de la r\'esolvante de Cayley ,
\item de Lagrange,
\item un polyn\^ome quelconque,
\end{enumerate}
alors chaque r\'esolvante associ\'ee est respectivement appel\'ee
{\it r\'esolvante }
\begin{enumerate}
\item {\it unitaire},
\item {\it lin\'eaire},
\item {\it altern\'ee},
\item {\it somme},
\item {\it sym\'etrique},
\item {\it produit},
\item {\it de Cayley},
\item {\it de Lagrange},
\item {\it g\'en\'erale}.
\end{enumerate}

Comme nous le verrons plus, il existe encore d'autres r\'esolvantes
(di\'edrale, de Klein, ...).

Nous d\'efinissons l'{\it arit\'e} d'une fonction comme le nombre de
variables qui apparaissent dans son expression. En g\'en\'eral si une fonction
est d'arit\'e $k$ nous notons $f(x_1,\ldots,x_k)$ \`a la place de
$f(x_1,\ldots ,x_k ,\ldots ,x_n)$.

Donnons des exemples pour chacun de ces cas :

\begin{enumerate}
\item $f(x)=x^7-x+1$ d'arit\'e 1,
\item $f(x_1,x_2) = x_1+3x_2$ d'arit\'e 2,
\item $f(x_1,x_2,x_3,x_4) = x_1 -x_2 + 3x_3-3x_4$ d'arit\'e 4,
\item $f(x_1,x_2,x_3) = x_1+x_2+x_3$ d'arit\'e 3,
\item $f(x_1,x_2,x_3) = 3x_1x_2 + 3x_2x_3 +3x_1x_3$ d'arit\'e 3,
\item $f(x_1,x_2) =x_1x_2$ d'arit\'e 2,
\item $f(x_1,x_2,x_3,x_4,x_5)=(x_1x2+x_2x_3+x_3x_4+x_4x_5+x_5x_1 -
        (x_1x_3+x_3x_5+x_5x_2+x_2x_4+x_4x_1))^2$
\item $f(x_1,x_2,x_3) = \epsilon x_1 + \epsilon^2 x_2 + \epsilon^3
x_3$, o\`u $\epsilon$ est une racine troisi\`eme de l'unit\'e,
\item $f(x_1,x_2,x_3) = x_1 +2x_2x_3$ d'arit\'e 3.
\end{enumerate}
Il est clair qu'une r\'esolvante somme ou produit est \'egalement
sym\'etrique et qu'une r\'esolvante somme ou altern\'ee est
\'egalement lin\'eaires.

Les calculs de ces r\'esolvantes se r\'ealisent de deux mani\`eres. Ou bien 
avec la fonction {\tt RESOLVANTE} ou bien avec un nom sp\'ecifique 
pr\'ec\'ed\'e de {\tt RESOLVANTE\_}. La liste des fonctions possibles est 
donc :

\begin{itemize}
\item {\tt RESOLVANTE\_PRODUIT\_SYM(p,x)} 
\index{RESOLVANTE\_PRODUIT\_SYM(p,x)} qui calcule la liste toutes les 
r\'esolvantes produit du polyn\^ome {\tt p(x)}
\item {\tt RESOLVANTE\_UNITAIRE(p,q,x)} 
  \index{RESOLVANTE\_UNITAIRE(p,q,x)}
qui calcule la r\'esolvante du 
polyn\^ome {\tt p(x)} par le polyn\^ome {\tt q(x)}
Soit $\prod_{p(\alpha)=0}(y-q(\alpha))$
\item {\tt RESOLVANTE\_ALTERNEE1(p,x)} 
  \index{RESOLVANTE\_ALTERNEE1(p,x)}
qui calcule la transformation de 
p(x) de degr\'e $n$ par la fonction $\prod_{1\leq i<j\leq n-1} (x_i-x_j)$
\item {\tt RESOLVANTE\_KLEIN(p,x)} 
  \index{RESOLVANTE\_KLEIN(p,x)}
qui calcule la transformation de
{\tt p(x)} par la fonction $x_1x_2+x_3$ et appel\'ee ainsi car elle
est associ\'ee au groupe de Klein,
\item {\tt RESOLVANTE\_KLEIN3(p,x)} qui calcule la transformation de
{\tt p(x)} par la fonction $x_1x_2x_4+x_4$
  \index{RESOLVANTE\_KLEIN3(p,x)}
\item {\tt RESOLVANTE\_VIERER(p,x)} 
  \index{RESOLVANTE\_VIERER(p,x)}
qui calcule la transformation de
{\tt p(x)} par la fonction $x_1x_2-x_3x_4$
\item {\tt RESOLVANTE\_DIEDRALE(p,x)} 
  \index{RESOLVANTE\_DIEDRALE(p,x)}
qui calcule la transformation de
{\tt p(x)} par la fonction $x_1x_2+x_3x_4$
  \index{RESOLVANTE\_BIPARTITE(p,x)}
 \index{RESOLVANTE\_BIPARTITE(p,x)} qui calcule la transformation de
{\tt p(x)} par la fonction $x_1x_2\ldots x_{n/2}+x_{n/2+1}\ldots x_n$.
Le degr\'e de $p$ doit n\'ecessairement \^etre pair
\item {\tt RESOLVANTE(p,x,f,[x1,x2,...,xk])} 
  \index{RESOLVANTE(p,x,f,[x1,x2,...,xk])}
qui calcule la transformation de
{\tt p(x)} par la fonction {\tt f} d'arit\'e {\tt k} et de variables 
{\tt x1,x2,...,xk} ; (L'arit\'e d'une fonction est le nombre minimum de 
variables n\'ecessaire pour l'\'ecrire).


\end{itemize}
Il est essentiel pour
l'efficacit\'e des calculs de ne mettre dans la liste des variables de
{\tt f} que celle qui apparaissent effectivement dans son expression.
Avant d'appeler la fonction {\tt resolvante}, selon le type de
r\'esolvante calcul\'ee on peut mettre un drapeau \`a la
variable {\tt resolvante}
afin d'utiliser l'algorithme le plus adapt\'e.
Selon les cas des r\'esolvantes cit\'es plus haut nous devrons mettre
le drapeau de la variable {\tt resolvante} respectivement \`a 
\begin{enumerate}
\item unitaire,
\item lineaire,
\item alternee,
\item somme,
\item symetrique,
\item produit,
\item Cayley
\item lagrange,
\item generale.
\end{enumerate}
On peut g\'en\'eraliser la notion de r\'esolvante par la
transformation de $r$ polyn\^ome par une fonction d\'ependant de
$r$ blocs de variables :
\begin{itemize}
\item {\tt DIRECT} \index{direct}
      ([$P_1,P_2,\ldots,P_r$],y,f,[$lvar_1,lvar_2,\ldots ,lvar_r$])
$\longrightarrow$ $f_*(P_1,P_2, \ldots, P_r)(y)$ \\
qui, \'etant donn\'e les $r$ polyn\^omes $P_1, \ldots, P_r$ en la variable
$y$ de 
degr\'es respectifs $d_1, \ldots , d_r$, ram\`ene
le polyn\^ome produit des $(y - h(a^{(1)}, \ldots, a^{(p)}))$ o\`u
$a^{(i)}$ est le $d_i$-uplet des racines de $P_i$ pour $i$ allant de 1
\`a $r$ et o\`u $h$ parcoure toute l'orbite de la fonction $f$ sous 
l'action du produit de groupes sym\'etriques
$S_{d_1}\times \cdots \times S_{d_r}$. Dans $lvar_i$ on met les variables
de $f$ qui apparaisent effectivement dans son expression. C'est \`a dire
que $lvar_i$ peut comporter moins de $d_i$ variables. La fonction
{\tt direct} se chargera d'en d\'eduire l'action du groupe
sym\'etrique $S_{d_i}$.
\end{itemize}
{\it Exemples}.
\small
\begin{verbatim}
  resolvante:unitaire;
  RESOLVANTE(x^7-14*x^5  + 56*x^3  - 56*X + 22,x,x^3-1,[x]);

        7      6        5         4          3   
      Y  + 7 Y  - 539 Y  - 1841 Y  + 51443 Y  

				  2
                        + 315133 Y  + 376999 Y + 125253  

 resolvante : lineaire$
 RESOLVANTE(x^4-1,x,x1+2*x2+3*x3,[x1,x2,x3]);
    24       20         16            12       
   Y   + 80 Y   + 7520 Y   + 1107200 Y   
                               8	      4
                   + 49475840 Y  + 344489984 Y + 655360000
                                                      
 resolvante : generale$
 RESOLVANTE(x^4-1,x,x1+2*x2+3*x3,[x1,x2,x3]);
    24       20         16            12       
   Y   + 80 Y   + 7520 Y   + 1107200 Y   
                               8	      4
                   + 49475840 Y  + 344489984 Y + 655360000

 RESOLVANTE(x^4-1,x,x1+2*x2+3*x3,[x1,x2,x3,x4]);
    24       20         16            12       
   Y   + 80 Y   + 7520 Y   + 1107200 Y   
                               8	      4
                   + 49475840 Y  + 344489984 Y + 655360000

 DIRECT([x^4-1],x,x1+2*x2+3*x3,[[x1,x2,x3]]);

    24       20         16            12       
   Y   + 80 Y   + 7520 Y   + 1107200 Y   
                               8	      4
                   + 49475840 Y  + 344489984 Y + 655360000

 resolvante_diedrale(x^5-3*x^4+1,x);
   15       12       11       10        9         8         7        6
  X   - 21 X   - 81 X   - 21 X   + 207 X  + 1134 X  + 2331 X  - 945 X

                     5          4          3          2
             - 4970 X  - 18333 X  - 29079 X  - 20745 X  - 25326 X - 697
\end{verbatim}
\normalsize
Nous constatons ainsi que la fonction {\tt DIRECT} est une g\'en\'eralisation
de la fonction r\'esolvante.
\small
\begin{verbatim} 
 resolvante : lineaire$
 RESOLVANTE(x^4-1,x,x1+2*x2,[x1,x2]);
                           12       8        4
                          Y   + 13 Y  + 611 Y  - 625
 DIRECT([x^4-1],x,x1+2*x2,[[x1,x2,x3]]);
                           12       8        4
                          Y   + 13 Y  + 611 Y  - 625
\end{verbatim}
\normalsize
Le r\'esultat de ce dernier calcul eut \'et\'e le m\^eme avec
\begin{verbatim} 
DIRECT([x^4-1],x,x1+2*x2,[[x1,x2]]);
\end{verbatim}
Comme de plus il est
plus efficace, il est conseiller de ne pas donner les variables 
n'intervenant pas dans l'expression de la fonction de transformation.
\small
\begin{verbatim} 
 resolvante:lineaire$
 RESOLVANTE(x^4-1,x,x1+x2+x3,[x1,x2,x3]);
                                 4
                                Y  - 1
 resolvante:symetrique$

 RESOLVANTE(x^4-1,x,x1+x2+x3,[x1,x2,x3]);
                                 4
                                Y  - 1
 resolvante:lineaire$
 RESOLVANTE(x^4+x+1,x,x1-x2,[x1,x2]);
               12      8       6        4        2
              Y   + 8 Y  + 26 Y  - 112 Y  + 216 Y  + 229
 resolvante:alternee$
 RESOLVANTE(x^4+x+1,x,x1-x2,[x1,x2]);
                12      8       6        4        2
              Y   + 8 Y  + 26 Y  - 112 Y  + 216 Y  + 229
 resolvante:generale$
 RESOLVANTE(x^4+x+1,x,x1-x2,[x1,x2]);
               12      8       6        4        2
              Y   + 8 Y  + 26 Y  - 112 Y  + 216 Y  + 229 
\end{verbatim}
\normalsize
Nous calculons ci-dessous une image directe de deux mani\`eres diff\'erentes.
La premi\`ere met en \'evidence les calculs interm\'ediaires
utilis\'es par la fonction {\tt DIRECT}.
On peut changer le drapeau de {\tt direct}. On a {\tt puissances} par
d\'efaut ce qui veut dire que l'on utilise la fonction {\tt MULTI\_PUI}. Si
on fait~: {\tt direct : elementaire}, la fonction {\tt DIRECT} utilise 
la fonction {\tt MULTI\_ELEM}
g\'en\'eralement moins performante.
\small
\begin{verbatim}

 l :  PUI_DIRECT(MULTI_ORBIT(a*x+b*y,[[x,y],[a,b]]),[[x,y],[a,b]],[2,2]);

                                    2  2
                 [a x, 4 a b x y + a  x ]

 m: MULTI_ELEM([[2,e1,e2],[2,f1,f2]],l[1],[[x,y],[a,b]]);

                           e1 f1

 n: MULTI_ELEM([[2,e1,e2],[2,f1,f2]],l[2],[[x,y],[a,b]]);

                            2             2     2   2
              8 e2 f2 - 2 e1  f2 - 2 e2 f1  + e1  f1

 PUI2ELE(2,[2,m,n]);

                                       2           2
              [2, e1 f1, - 4 e2 f2 + e1  f2 + e2 f1 ]
 
 ELE2POLYNOME(%,y);

                    2                         2           2
                   y  - e1 f1 y - 4 e2 f2 + e1  f2 + e2 f1

 DIRECT([z^2  - e1* z + e2, z^2  - f1* z + f2], z, b*v + a*u, 
              [[u, v], [a, b]]);

                    2                         2           2
                   y  - e1 f1 y - 4 e2 f2 + e1  f2 + e2 f1

\end{verbatim}
\normalsize
\begin{itemize}
\item{\tt PUI\_DIRECT($[f_1, \ldots, f_q]$,[$lvar_1,\ldots ,lvar_p$],
[$d_1,d_2,...,d_p$])}\\
\index{pui\_direct}
 Soit $f$ un polynome en $r$ blocs de variables $lvar_1,\ldots ,lvar_r$
  Soit $c_i$ le nombre de variables dans $lvar_i$  et $S_C$ le produit des $r$
  groupes sym\'etriques $S_{c_i}$ de degr\'es respectifs
  $c_1,...,c_r$. Ce groupe agit
  naturellement sur $f$.
  La liste {\tt ORBITE} est l'orbite, not\'ee $S_Cf$, de la fonction $f$ 
  sous  l'action de $S_C$. (Cette liste peut \^etre obtenue avec la fonction 
  {\tt MULTI\_ORBIT}).
  Les $d_i$ sont des entiers tels que 
  $c_1\leq d_1\;, c_2 \leq d_2 \;,\ldots ,c_p\leq d_p$.
  Notons $S_D$ le produit des groupes sym\'etriques 
$S_{d_1} \times S_{d_2} \times \ldots \times S_{d_p}$.\\

  La fonction {\tt PUI\_DIRECT} ram\`ene les $N$ premi\`eres fonctions 
  puissances de l'orbite $S_Df$ de la fonction $f$
  d\'eduites des fonctions puissances de l'orbite $S_Cf$ o\`u 
  $N$ est le cardinal de $S_Df$.

  Le r\'esultat est rendu sous forme multi-contract\'ee par rapport \`a $S_D$.
 (i.e. on ne conserve qu'un \'el\'ement par orbite sous l'action de $S_D$).
\end{itemize}
\small
\begin{verbatim}
 L:[[x,y],[a,b]]$

 pui_direct ([b*y + a*x, a*y + b*x],L,[2,2]);

                                2  2
             [a x, 4 a b x y + a  x ]

  pui_direct([b*y + a*x, a*y + b*x], L,[3,2]);

                         2  2     2    2        3  3
  [2 A X, 4 A B X Y + 2 A  X , 3 A  B X  Y + 2 A  X ,

        2  2  2  2      3    3        4  4
    12 A  B  X  Y  + 4 A  B X  Y + 2 A  X ,

        3  2  3  2      4    4        5  5
    10 A  B  X  Y  + 5 A  B X  Y + 2 A  X ,

        3  3  3  3       4  2  4  2      5    5        6  6
    40 A  B  X  Y  + 15 A  B  X  Y  + 6 A  B X  Y + 2 A  X ]

 pui_direct ([y+x+2*c, y+x+2*b, y+x+2*a],[[x,y],[a,b,c]],[2,3]);

                             2              2
      [3 x + 2 a, 6 x y + 3 x  + 4 a x + 4 a , 

              2                   3        2       2        3
           9 x  y + 12 a x y + 3 x  + 6 a x  + 12 a  x + 8 a ]


\end{verbatim}
\normalsize
Evidemment, on retrouve le m\^eme r\'esultat ainsi 
\small
\begin{verbatim}

 pui_direct([y+x+2*a],[[x,y],[a]],[2,3]);

                             2              2
      [3 x + 2 a, 6 x y + 3 x  + 4 a x + 4 a , 

              2                   3        2       2        3
           9 x  y + 12 a x y + 3 x  + 6 a x  + 12 a  x + 8 a ]
\end{verbatim}
\normalsize
\newpage
\section*{Signification des objets }

{\tt card}
 est le cardinal de l'ensemble des variables sur lequel on travaille.\\

{\tt $e_i$} : i\`{e}me fonction sym\'{e}trique \'{e}l\'{e}mentaire.\\

{\tt $p_i$} : i\`{e}me fonction puissance.\\

{\tt $h_i$} : i\`{e}me fonction compl\`ete.\\

{\tt ele} = [$e_{1},e_{2},e_{3},...,e_{n}$] , $n$ intervenant dans le
descriptif des fonctions.\\

{\tt cele} = [card,$e_{1},e_{2},e_{3},...,e_{n}$]\\

{\tt pui} = [$p_{1},p_{2},p_{3},...,p_{m}$] , $m$ intervenant dans le
descriptif des fonctions.\\

{\tt cpui} = [card,$p_{1},p_{2},p_{3},...,p_{m}$].\\

{\tt ccomp} = [card,$h_{1},h_{2},h_{3},...,h_{m}$].\\

{\tt sym} :  polyn\^{o}me sym\'{e}trique sans pr\'{e}cision sur
sa repr\'{e}sentation.\\

{\tt fmc} :  forme monomiale contract\'{e}e.\\

{\tt part} :  partition.\\

{\tt tc} :  terme contract\'{e}.\\

{\tt tpart} : terme partitionn\'{e}.\\

{\tt psym} : polyn\^{o}me sym\'{e}trique sous sa forme \'etendue.\\

{\tt pc} : polyn\^{o}me sym\'etrique sous une forme contract\'{ee}.\\

{\tt multi\_pc} : polyn\^{o}me multisym\'etrique sous une forme
multicontract\'{ee} sous $S_D$.\\

{\tt ppart} : polyn\^{o}me sym\'etrique sous sa forme partitionn\'{ee}.\\

{\tt P($x_1, \ldots , x_q$)} :  polyn\^{o}me en $x_1, \ldots , x_q$.\\

{\tt lvar} : liste de variables.\\

$[lvar_1, \ldots,lvar_r]$ : liste de listes de variables.

\section*{Index des fonctions disponibles}
\begin{tabular}{cc}
\indexentry{arite(degre, arite, puissances)}{4}
\indexentry{card\_orbit(partition,n)}{5}
\indexentry{card\_stab(l,egal)}{5}
\indexentry{comp2ele(n, ccomp) }{11}
\indexentry{comp2pui(n, ccomp)}{11}
\indexentry{cont2part(pc,lvar)}{6}
\indexentry{contract(polyn\^ome,lvar)}{6}
\indexentry{direct}{15}
\indexentry{ele2comp(m , cele)}{10}
\indexentry{ele2polynome(cele,z)}{5}
\indexentry{ele2pui(m,cele) }{9}
\indexentry{elem(cele,sym,lvar)}{8}
\indexentry{explose(pc,lvar) }{6}
\indexentry{kostka(part1,part2)}{7}
\indexentry{lgtreillis(n,m)}{7}
\indexentry{ltreillis(n,m)}{7}
\indexentry{mon2schur(liste) }{11}
\indexentry{multi\_elem([cele$_{1}$, \ldots, cele$_{p}$],multi\_pc, [lvar$_{1}$, \ldots,lvar$_{r}$])  }{8}
\indexentry{multi\_orbit(polyn\^ome,[lvar$_{1}$,lvar$_{2}$,\ldots ,lvar$_{r}$])}{7}
\indexentry{multi\_pui([cpui$_{1}$, \ldots, cpui$_{p}$],multi\_pc, [lvar$_{1}$, \ldots,lvar$_{p}$]) }{8}
\indexentry{multinomial(r part)}{5}
\indexentry{multsym(ppart1, ppart2,n)}{8}
\indexentry{orbit(polyn\^ome,lvar) }{7}
\indexentry{part2cont(ppart,lvar)}{6}
\indexentry{partpol(polyn\^ome, lvar)}{6}
\indexentry{permut(l)}{5}
\indexentry{polynome2ele(polyn\^ome,z)}{5}
\indexentry{pui(cpui,sym,lvar) }{8}
\indexentry{pui2comp(n, cpui) }{10}
\indexentry{pui2ele(n,cpui) }{10}
\indexentry{pui\_direct}{17}
\indexentry{puireduc(n,cpui) }{10}
\indexentry{resolvante(p,x,f,[x1,x2,...,xk])}{14}
\indexentry{resolvante\_alternee1(p,x)}{14}
\indexentry{resolvante\_bipartite(p,x)}{14}
\indexentry{resolvante\_diedrale(p,x)}{14}
\indexentry{resolvante\_klein(p,x)}{14}
\indexentry{resolvante\_klein3(p,x)}{14}
\indexentry{resolvante\_produit\_sym(p,x)}{14}
\indexentry{resolvante\_unitaire(p,q,x)}{14}
\indexentry{resolvante\_vierer(p,x)}{14}
\indexentry{schur2comp(p,l) }{11}
\indexentry{tcontract(polyn\^ome,lvar)}{6}
\indexentry{tpartpol(polyn\^ome,lvar)}{6}
\indexentry{treillis(n)}{7}
\indexentry{treinat(part)}{7}
\end{tabular}
\tableofcontents 
\end{document}
